% \documentclass{article}
% \usepackage[utf8]{inputenc}

% \title{ee547_final_report}
% \author{Ben Guo}
% \date{December 2021}

\documentclass[11pt]{article}
\usepackage{graphicx}
\usepackage{amssymb}
\usepackage{amsmath}
\usepackage{enumitem}

\usepackage[html,dvipsnames]{xcolor}
% \usepackage[
%         bibencoding=utf8, 
%         style=alphabetic
%     ]{biblatex}


\setlength{\textwidth}{6.5in}
\setlength{\textheight}{9.0in}
\headheight=0.5in
\topmargin=-0.5in
\oddsidemargin= 0.0in
\evensidemargin=-0.25in


\usepackage[pdfauthor={Brandon Franzke},pdftitle={EE 541 Project Proposal},% 
pdftex,bookmarks]{hyperref} 
\hypersetup{colorlinks,% 
citecolor=green,% 
filecolor=Orange,% 
linkcolor=blue,% 
urlcolor=BrickRed,% 
pdftex} 



\pagestyle{myheadings}
\markright{{\bf EE 547 - Haoyu Guo, Ziyou Geng and Zhenyang Li - Fall 2021} }
\title{\bf {\small EE 547 -- Applied and Cloud Computing} \\ Project Report \\ \quad \\ BufferOverflow: Mini StackOverflow Website}
\author{Haoyu Guo, Ziyou Geng and Zhenyang Li}

\begin{document}
\maketitle
\newpage
\tableofcontents
\newpage

\section{Objective}

This project's objective is to provide an online platform where users can post questions and answer questions. Apart from this, the website can label tags and show popular tags, give every answer a vote by the users. We create this website to mimic StackOverflow functionalities, which is our most frequently used website. 

% It is composed on two main sections, the front-end and back-end with RESTful API. The details are as follows.  
% Music Streaming Platform


\section{Project Architecture}
This project consist of a client-side (Front-end) and a server-side (Back-end), plus MongoDB database and publishing services. Both the Front-end and the Back-end will adopt MVC pattern. The main technology stack is Vue.js + Node.js with Express + MongoDB with Mongoose. 

\begin{figure}[htp]
    \centering
    \includegraphics[width=10cm]{image1}
    \caption{Overall Architecture}
    \label{fig:galaxy}
\end{figure}


\subsection{Front-end Architecture}

For the Front-end part, we utilize Vue.js 3.0 with Vue Router and VueX. 
Vue Router is a convenient routing tool to switch between different pages in a Vue project. Vuex can store data globally, so that each page can access the data in Vuex and we do not have to constantly send data between components. In our project, Vuex is used to store user's login information. In addition, Ant-Design-Vue UI library is implemented to expedite and simplify the development process. The project will be initialized by Vue CLI. Based on this, rich text editor module will be implemented, sign-up, sign-in and authentication management will be realized based on JWT and all requests will be sent with Axios. 

\subsubsection{Client Structure}
\begin{figure}[htp]
    \centering
    \includegraphics[width=10cm]{client}
    \caption{Client Structure}
    \label{fig:galaxy}
\end{figure}

Figure.2 is the structure of our client-side. Router folder contains routes to each page. Views folder contains all the pages. For the module to be reused, such as header, footer and question, we  encapsulated them as components. The store folder contains configuration file of Vuex.

% \begin{description}
%   \item[Home Page] 
%   \item[User Page]
%   \item[Question Page]
% \end{description}
\subsubsection{Home Page}
\begin{figure}[htp]
    \centering
    \includegraphics[width=11cm]{home}
    \caption{Home Page}
    \label{fig:galaxy}
\end{figure}
For home page, we encapsulated a common components of a single question card and added v-for to it to create a question list
The right sider of this page is currently popular tags. The left sider contains The navigation bar on the left is the site's three main features, but some of them are not yet implemented. This page also has a button to publish new question.

\subsubsection{Signin Page}
\begin{figure}[htp]
    \centering
    \includegraphics[width=10cm]{user_login}
    \caption{Signin Page}
    \label{fig:galaxy}
\end{figure}
Both the server-side and client-side have authentication mechanisms for the sign-in operation. If user do not input username or password, or input a short one, a message will pop up and the web page will not send any requests to server. Users can also switch to sign-up page, the page automatically sends a sign-in request after they register, so users don't have to type twice.

\subsubsection{Question Posting Page}
\begin{figure}[htp]
    \centering
    \includegraphics[width=7.91cm]{question_post}
    \caption{Question Posting Page}
    \label{fig:galaxy}
\end{figure}
The Title and Tags are text that would be sent to server-side as \textit{string}. For the body part, we utilized a rich text editor called \textit{Wangeditor}. \textit{Wangeditor}is very powerful, and user can insert code block, image, and even video into it. The raw data of body is \textit{html} file and we convert it into \textit{String}, making it much easier to store in our database.

\subsubsection{Question Answering Page}
\begin{figure}[htp]
    \centering
    \includegraphics[width=10cm]{question_answer}
    \caption{Question Detail Page}
    \label{fig:galaxy}
\end{figure}
The question detail page contains three feature: display detail of a question, display answer of this question, Edit and publish new answer about this question. The answer module contains a delete button, which appears only when the author of this answer is the user signed in at this computer. It also contains a vote button and a unvote button, When a user votes on an answer, the vote button is updated but the page is not refreshed. The answer edit module is also a rich text editor.

\subsubsection{Search Result Page}
\begin{figure}[htp]
    \centering
    \includegraphics[width=11cm]{react}
    \caption{Search result Page}
    \label{fig:galaxy}
\end{figure}
We can use the searching bar at the header to search questions of interest. This page is similar to home page but it only contains questions that match the search keywords. 

\newpage

\subsection{Back-end Architecture}
For the Back-end part, we utilize Express and Mongoose. RESTful API is adopted and CRUD operations will be conducted by Mongoose. We also implement user sign-up and authentication, text editing, ask questions, answer questions and vote for each answer. 

\begin{description}
  \item[Sign-up endpoint]  Use HTTP post('/signup') and check whether username and password can meet our requirement and cannot be the same with former users
  \item[Authentication endpoint] Use HTTP post('/authenticate') and check username and password with data in MongoDB database
%   \item[Sign-up] Join as an artist or user.  Gather basic info such as email, unique username, password, and preferred music genres
  \item[Show Homepage endpoint] Use HTTP get('/tags') and get('/question'), to make sure we can get the recent asked questions and popular tags from the home page
  \item [Ask Question endpoint] Use HTTP post('/questions') and check with the user authentication, check whether title, text and tags meet our requirement
  \item [Answer Question endpoint]  use HTTP post('/answer/:question') and check whether text meet our answer requirement
  \item [Remove Answer endpoint] use HTTP delete('/answer/:question/:answer') and check whether the user who want to delete is the one who write this answer. If yes, then delete this answer. 
  \item [Vote Answer endpoint] use HTTP get('/votes/upvote/') and get('/votes/downvote'), to make sure we can get how many votes each answer has. 
\end{description} 

\newpage
\section{Implementation}

\subsection{User Micro-services}
The micro-service component handles mainly: user registration, authentication and login. 

Node.js with express-validator are used to hanlde user registration and user authentication. All user data is stored in a document using MongoDB. All users need to register their account first, after registered, the user is redirected to the login page. 

\subsubsection{User Schema}
After successfully registered, the user can login to their account and access his/her profile. The user schema with all fields are shown below in figure. 
\begin{figure}[htp]
    \centering
    \includegraphics[width=11cm]{user}
    \caption{User Schema in MongoDB}
    \label{fig:galaxy}
\end{figure}


\subsubsection{User Authentication}
JSON Web Token(JWT) will be used for authentication and authorization. jsonwebtoken package will help to set up protected routes that only logged in users can access.

The user registration route creates a new user with the given information that we access from the req.body. After saving the user, we generate an authentication token and return it as a response alongside the user data. 
We can test our endpoint to see whether it works or not in Postman. 
\begin{figure}[htp]
    \centering
    \includegraphics[width=12cm]{user2}
    \caption{User Authentication in Postman}
    \label{fig:galaxy}
\end{figure}

\newpage
\subsection{Question Handling Micro-services}
Question handling micro-services is the major part of this website implementation. We include all the json data apart from user information in this part. 
\subsubsection{Question Schema}
We have two Schema in the MongoDb collection. Apart from the User schema, there is the Questions schema. Answer model and vote model both pass the data to Questions schema. So we can see all the data in json format in Questions schema. 

\begin{figure}[htp]
    \centering
    \includegraphics[width=12cm]{question}
    \caption{Question Schema in MongoDB}
    \label{fig:galaxy}
\end{figure}


\subsection{API Design Implementation}

\begin{figure}[htp]
    \centering
    \includegraphics[width=10cm]{api}
    \caption{API Design Implementation}
    \label{fig:galaxy}
\end{figure}



\newpage

\section{Host On Cloud Service}
% We anticipate a single t2.small server will suffice.
% We will use AWS Simple Queue Service (SQS) and Simple Notifiation Service (SNS) to control the async processing.
% We don't anticipate any unusual hardware or compute needs.

\subsection{Introduction to AWS and Hosting Service}
\begin{figure}[htp]
    \centering
    \includegraphics[width=10cm]{aws.png}
    \caption{AWS Host Website}
    \label{fig:galaxy}
\end{figure}

AWS (Amazon Web Services) is a comprehensive, evolving cloud computing platform provided by Amazon that includes a mixture of infrastructure, through which we could do scalable and cost-effective cloud. We could do file and data storage and backup, several web applications or even host our own website on the AWS cloud. The service is famous for its high security, global availability and flexibility.

Elastic Compute Cloud (EC2) provides instances as virtual servers for cloud computing capacity. By using EC2, we can rent the virtual server and use it to deploy our own applications.


\subsection{Configuration of Service}
We choose using AWS to reduce our burden of maintenance and deployment since those tools could help us with instance running our back-end program and deal with the connection to front-end.

We create a new private VPC in the AWS cloud. The data would be stored in the RDS. For instance launching, we just choose Amazon Machine Image (AMI) and EC2 Instance Type, configure instance, add storage, configure security groups and then we could start our own services.

\subsection{Instances Parameters}
Check the parameters we use in our instance in Table  \ref{table:1} 

\begin{table}[h!]
\centering
\begin{tabular}{|c|c|}
    \hline
    Parameters & Content \\ \hline\hline
    Instance ID   & i-02b8a5bdd35ed6cf6 \\ \hline
    Hostname type & IP name: ip-172-31-8-253.us-west-1.compute.internal \\ \hline
    Instance type & t2.micro \\ \hline
    Public IPv4 address & 54.193.232.33\\ \hline
    VPC ID & vpc-01a6378edf9b586cf \\ \hline
    Subnet ID & subnet-0e8eeea11d94f6ab0 \\ \hline
    Platform &  Ubuntu (Inferred) \\ \hline
    Platform details &  Linux/UNIX \\ \hline
    AMI ID &  ami-053ac55bdcfe96e85 \\ \hline
    AMI name &  ubuntu/images/hvm-ssd/ubuntu-focal-20.04-amd64-server-20211021\\ \hline
    Reservation & r-0ce2bc369236b1019\\ \hline
    Virtualization type &  hvm \\ \hline
    Number of vCPUs & 1 \\ \hline
    Security groups & sg-08823c7df264ecae6 (launch-wizard-1) \\ \hline
    Volume ID & vol-0427ed3b2e43cb690 \\ \hline
    Device name & /dev/sda1 \\ \hline
    Volume size (GiB) & 8 \\ \hline
    
\end{tabular}
\caption{Parameters of EC2 Instances}

\label{table:1}
\end{table}

\newpage
\section{Datasources and Additional APIs}
Users can post their own questions and the answers will be posted by other users. Mongoose schema related to questions, answers and users will be stored in MongoDB database. 


\begin{figure}[htp]
    \centering
    \includegraphics[width=12cm]{terminal}
    \caption{MongoDB collections}
    \label{fig:galaxy}
\end{figure}




\section{Summary and Conclusion}
In this project, we propose to build a coding forum website similar to StackOverflow. 

There are two main challenges we face. First, we are not familiar with Vue.js and have to start it all from scratch. Second, we have very limited full stack programming experience and debugging cost us a lot of time.

Finally, we learn a lot from this class EE547 and this mini StackOverflow website project is far from perfect. We still need to better our UI design and add more Back-end and Front-end features. 
% Users can post all kinds of questions about programming to the platform and can also get answers from other users. Specifically, the website will have a question publishing module, a question list module and a question content module with functionalities for deleting content. The platform will also have a complete user management system, including user registration and login, authority management.
% After development, We deploy the whole project to AWS.  



\section{Team Roles} 

The following is the rough breakdown of roles and responsibilities we plan for our team:
\begin{itemize}
\item Haoyu Guo: Primary: Back-end develop and API devise, MongoDB resolvers and database access.  Secondary: assist with some layout elements.
\item Ziyou Geng: Primary: Front-end design. Secondary: Assist in API devise, ensure that client-side requests and Back-end API are compatible.
% \item Zhenyang Li: Client UX and UI.  Primary: build client side interface for API and create rough pages with dynamic server content.  Secondary: assist with design and styling
\item Zhenyang Li: Primary: Design test cases for the server-side using Jest. Secondary: host the website in AWS
\end{itemize}
All team members will work on the final presentation, slides, and report. 



\section{References, Tutorials, Codebases, Documentation, and Libraries}  
\begin{itemize}
    \item \noindent [1] wangeditor: a useful rich text editor module. https://github.com/awamwang/vue-wangeditor-awesome
    \item \noindent [2] Design and Implementation of a Vue.js-Based College Teaching System
    \item \noindent [3] Vue 3.0 official documentation: https://v3.vuejs.org/api/
    \item \noindent [4] https://stackoverflow.com/
    \item \noindent [5] StackOverflow API: https://stackoverflow-clone-dev. herokuapp.com/api/docs/
    \item \noindent [6] Stackexchange API: https://api.stackexchange.com/docs
    \item \noindent [7] Jest Official documentation:  https://jestjs.
    \item \noindent [8] AWS Service Introduction https://www.simplilearn.com/tutorials/aws-tutorial/what-is-aws

\end{itemize}



 \end{document}
